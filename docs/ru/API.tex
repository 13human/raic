\chapter{Описание API}\label{api}

\section{MyStrategy}

В языковом пакете для вашего языка программирования находится файл с именем \texttt{MyStrategy/my\_strategy.<ext>}.
В нем находится класс \texttt{MyStrategy}, в котором содержится метод \texttt{act}, содержащий логику вашей стратегии.

Этод метод будет вызываться каждый тик, для каждого вашего робота отдельно.

\begin{verbatim}
class MyStrategy:
      method act(me: Robot, rules: Rules, game: Game, action: Action):
            // Your implementation
\end{verbatim}

Параметры в методе \texttt{act}:
\begin{itemize}
      \item \texttt{me} --- ваш робот, действие которого сейчас определяется
      \item \texttt{rules} --- объект, содержащий правила игры (не меняется между тиками)
      \item \texttt{game} --- объект, содержащий информацию о текущем состоянии игры (меняется между тиками)
      \item \texttt{action} --- объект, задающий действие вашего робота. Меняя поля этого объекта, вы задаете действие
\end{itemize}

\section{Action}

Объект, определяющий действие робота.

\begin{verbatim}
class Action:
      target_velocity_x: Float      // Координата X желаемого вектора скорости
      target_velocity_y: Float      // Координата Y желаемого вектора скорости
      target_velocity_z: Float      // Координата Z желаемого вектора скорости
      jump_speed: Float             // Желаемая скорость прыжка
      use_nitro: Bool               // Использовать/не использовать нитро
\end{verbatim}

\section{Arena}

Объект, определяющий размеры арены (см. \ref{arena_form}).

\begin{verbatim}
class Arena:
      width: Float
      height: Float
      depth: Float
      bottom_radius: Float
      top_radius: Float
      corner_radius: Float
      goal_top_radius: Float
      goal_width: Float
      goal_height: Float
      goal_depth: Float
      goal_side_radius: Float
\end{verbatim}

\section{Ball}

Объект, определяющий мяч.

\begin{verbatim}
class Ball:
      x: Float                      // Текущие координаты центра мяча
      y: Float
      z: Float
      velocity_x: Float             // Текущая скорость мяча
      velocity_y: Float
      velocity_z: Float
      radius: Float                 // Радиус мяча
\end{verbatim}

\section{Game}

Объект, содержащий себе информацию о текущем состоянии игры.

\begin{verbatim}
class Game:
      current_tick: Int             // Номер текущего тика
      players: Player[]             // Список игроков
      robots: Robot[]               // Список роботов
      nitro_packs: NitroPack[]      // Список рюкзаков с нитро
      ball: Ball                    // Мяч
\end{verbatim}

\section{NitroPack}

Объект, определяющий рюкзак с нитро.

\begin{verbatim}
class NitroPack:
      id: Int
      x: Float                      // Координаты центра рюкзака
      y: Float
      z: Float
      radius: Float                 // Радиус
      respawn_ticks: Int?           // Кол-во тиков, через которое рюкзак снова появится
                                          (или null, если еще не был подобран)
\end{verbatim}

\section{Player}

Объект, определяющий игрока (стратегию).

\begin{verbatim}
class Player:
      id: Int
      me: Bool                      // true, если это оъект вашего игрока
      strategy_crashed: Bool
      score: Int                    // Текущий счет игрока
\end{verbatim}

\section{Robot}

Объект, определяющий робота.

\begin{verbatim}
class Robot:
      id: Int
      player_id: Int
      is_teammate: Bool             // true, если это робот вашего игрока
      x: Float                      // Текущие координаты центра робота
      y: Float
      z: Float
      velocity_x: Float             // Текущая скорость робота
      velocity_y: Float
      velocity_z: Float
      radius: Float                 // Текущий радиус робота
      nitro_amount: Float           // Текущий запас нитро робота
      touch: bool                   // true, если робот сейчас касается арены
      touch_normal_x: Float         // Вектор нормали к точке соприкосновения с ареной
      touch_normal_y: Float               (или null, если нет касания)
      touch_normal_z: Float
\end{verbatim}

\section{Rules}

Объект, задающий данные о мире, не меняющиеся между тиками. Также содержит константы симулятора

\begin{verbatim}
class Rules:
      max_tick_count: Int           // Максимальная продолжительность игры
      arena: Arena                  // Описание размеров арены
      team_size: Int                // Количество роботов в каждой команде

      // Константы, используемые в симуляторе
      ROBOT_MIN_RADIUS: Float
      ROBOT_MAX_RADIUS: Float
      ROBOT_MAX_JUMP_SPEED: Float
      ROBOT_ACCELERATION: Float
      ROBOT_NITRO_ACCELERATION: Float
      ROBOT_MAX_GROUND_SPEED: Float
      ROBOT_ARENA_E: Float
      ROBOT_RADIUS: Float
      ROBOT_MASS: Float
      TICKS_PER_SECOND: Int
      MICROTICKS_PER_TICK: Int
      RESET_TICKS: Int
      BALL_ARENA_E: Float
      BALL_RADIUS: Float
      BALL_MASS: Float
      MIN_HIT_E: Float
      MAX_HIT_E: Float
      MAX_ENTITY_SPEED: Float
      MAX_NITRO_AMOUNT: Float
      START_NITRO_AMOUNT: Float
      NITRO_POINT_VELOCITY_CHANGE: Float
      NITRO_PACK_X: Float
      NITRO_PACK_Y: Float
      NITRO_PACK_Z: Float
      NITRO_PACK_RADIUS: Float
      NITRO_PACK_AMOUNT: Float
      NITRO_PACK_RESPAWN_TICKS: Int
      GRAVITY: Float
\end{verbatim}